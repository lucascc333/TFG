\appendix % Inicio sección de apéndices

\chapter{Series formales} \label{series formales} % Crea el título sin numerar "Apéndices"


%: SERIES FORMALES
%%%%%%%%%%%%%%%%%%
%%%%%%%%%%%%%%%%%%
%%%%%%%%%%%%%%%%%%
%%%%%%%%%%%%%%%%%%

Toda la información de esta sección ha sido extraída de \cite{sambale}. Damos aquí las definiciones y los resultados de los que nos servimos a lo largo de todo el trabajo.

\begin{definicion}
    Sea $\mathcal{R}$ un dominio íntegro y sea $\alpha$ una sucesión infinita en $\mathcal{R}$, es decir:
    \begin{equation*}
        \alpha = \{a_0, a_1, a_2, \dots \} \hspace{0.25cm} a_n \in \mathcal{R}, \forall n \in \N 
    \end{equation*}
    Se define y simboliza con $\F$ al conjunto de estas sucesiones. Se dice que dos elementos $\alpha = \{a_0, a_1, a_2, \dots \}$ y $\beta = \{b_0, b_1, b_2, \dots \} $ de $\F$ son iguales si y solo si $a_n = b_n$ para todo $n$. 
\end{definicion}
Podemos dotar a este conjunto de una estructura algebraica mediante las operaciones de suma y producto definidas como:
\begin{equation*}
    \alpha + \beta \equiv \{a_0 + b_0, a_1 + b_1, a_2 + b_2, \dots \}
\end{equation*}
\begin{equation*}
    \alpha \cdot \beta \equiv \{a_0  b_0, a_0b_1 + a_1b_0, a_0b_2 + a_1b_1 + a_2b_0, \dots, \sum_{k=0}^na_kb_{n-k}, \dots\}
\end{equation*}

\begin{teorema}
    $(\F, +, \cdot)$ es un anillo. El elemento neutro y la unidad son, respectivamente:
    \begin{equation*}
        0 \equiv \{0,0,0,\dots\} \hspace{2cm} 1\equiv \{1, 0, 0, \dots\}
    \end{equation*}
    y el elemento opuesto o inverso respecto a la suma es:
    \begin{equation*}
        -\alpha \equiv \{-a_0, -a_1, -a_2, \dots \}
    \end{equation*}
\end{teorema}

\begin{definicion}
    Se definen los subconjuntos $\F_0$ y $\F_1$ de $\F$ como:
    \begin{equation*}
        \F_0 \equiv \{ \alpha \in \F : a_0 = 0 \}
    \end{equation*}
    \begin{equation*}
        \F_1 \equiv \{ \alpha \in \F : a_0 = 1 \}
    \end{equation*}
\end{definicion}

\begin{teorema}
    \textbf{(existencia inverso multiplicativo)} Un elemento $\alpha = \{a_0, a_1, a_2, \dots\} \in \F$ tiene inverso multiplicativo si y solo si $a_0 \in \mathcal{R}^*$. En particular, si $\mathcal{R}$ es un cuerpo, $\alpha$ tiene inverso multiplicativo si y solo si $\alpha \not \in \F_0$ 
\end{teorema}

\begin{proposicion}
    Dado un entero positivo $n$ y $\alpha \in \F_1$, se verifica que $(\alpha^{-1})^n = (\alpha^n)^{-1}$ 
\end{proposicion}

\begin{definicion}
    Se llama \textbf{anillo de series formales en la variable $T$ con coeficientes en $\mathcal{R}$} al conjunto $\mathcal{R}[[T]]$ definido como:
     \begin{equation*}
         \mathcal{R}[[T]] \equiv \{ F(T) = \sum_{n \geq 0} a_nT^n : a_n \in \mathcal{R}\}
    \end{equation*}
    dotado de las operaciones de suma y producto dadas por:
   \begin{equation*}
    F(T) + G(T) \equiv \sum_{n\geq 0} (a_n + b_n)T^n  
    \end{equation*}
    \begin{equation*}
    F(T) \cdot G(T) \equiv \sum_{n \geq 0}c_nT^n \text{ donde } c_n=\sum_{k=0}^na_nb_{n-k}
    \end{equation*}
    En concreto, a la variable $T$ se la denomina la \textbf{variable formal}
\end{definicion}

Con las definiciones dadas para la suma y el producto es natural que pensemos y escribamos los elementos de $\F$ como elementos de $\mathcal{R}[[T]]$:
\begin{equation*}
    \alpha = \{a_0, a_1, a_2, \dots \} \iff \alpha \equiv F(T) = \sum_{n=0}^\infty a_nT^n
\end{equation*}
ya que el producto y suma de los factores de $\alpha$ coincide con el producto y suma de cada factor de la serie formal $F(T)$

%%%%%%%%%%%%%%%%%%
%%%%%%%%%%%%%%%%%%
%%%%%%%%%%%%%%%%%%
%%%%%%%%%%%%%%%%%%
%%%%%%%%%%%%%%%%%%