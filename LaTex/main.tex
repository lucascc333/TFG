%!TEX encoding = UTF-8 Unicode
\documentclass[a4paper,12pt,centertags,intlimits,twoside]{report}

%: PAQUETES DE LATEX
%%%%%%%%%%%%%%%%%%%%%%%%%%%%%%%%%%%%
%%%%%%%%%%%%%%%%%%%%%%%%%%%%%%%%%%%%
%%%%%%%%%%%%%%%%%%%%%%%%%%%%%%%%%%%%
%%%%%%%%%%%%%%%%%%%%%%%%%%%%%%%%%%%%
%%%%%%%%%%%%%%%%%%%%%%%%%%%%%%%%%%%%
%%%%%%%%%%%%%%%%%%%%%%%%%%%%%%%%%%%%

%: Paquetes básicos
\usepackage[utf8]{inputenc} % Para la codificación unicode
\usepackage[T1]{fontenc} %Para la codificación unicode
\usepackage[spanish]{babel} % Para trabajar en castellano
\usepackage{enumitem} % Para enumerar casos
\usepackage{graphicx} % Para insertar imágenes
\usepackage{index} % Para el índice
\usepackage{csquotes} % Para las citas de la bibliografía
\usepackage{pdfpages} % Para cargar pdf's
\usepackage{parskip} % Para eliminar la sangría


%: Paquetes de estilo
\usepackage[left=2.5cm,right=2.5cm,top=3.1cm,bottom=3.1cm]{geometry} % Para definir los márgenes y el tipo de papel
\usepackage{ragged2e} % Para alinear texto
\usepackage{titlesec} % Para cambiar el estilo de los títulos
\usepackage[x11names,dvipsnames,svgnames]{xcolor}% Para utilizar colores
\usepackage{fancyhdr} % Para el estilo del encabezado
\usepackage{hyperref} % Para poder clickar links
\hypersetup{
    colorlinks = true, linkcolor = {DarkRed}, urlcolor={Cyan}, citecolor = {DarkRed}
} 
\usepackage{lettrine} % Para añadir letras grandes al comienzo de un párrafo

%: Paquetes matemáticos
\usepackage{amsfonts} % Añade letras góticas, negritas... etc
\usepackage{amsmath, amssymb, amsthm} % Para alinear fórmulas y ecuaciones, añadir símbolos básicos, escribir teoremas y demos...
    % Esto nos deja decir \begin{demo} en vez de \begin{proof}
    \let\demo\proof
    \let\enddemo\endproof
\usepackage{mathtools} % Para facilitar la escritura de matemáticas
\usepackage{mathrsfs} % Añade estilos para escribir símbolos
\usepackage{stmaryrd} % Añade más símbolos
\usepackage{cases} % Para distinguir casos al definir una función
\usepackage{framed} % Para recuadrar fórmulas y texto
\usepackage{tikz-cd} % Para crear diagramas con flechas
\usetikzlibrary{babel}

%%%%%%%%%%%%%%%%%%%%%%%%%%%%%%%%%%%%
%%%%%%%%%%%%%%%%%%%%%%%%%%%%%%%%%%%%
%%%%%%%%%%%%%%%%%%%%%%%%%%%%%%%%%%%%
%%%%%%%%%%%%%%%%%%%%%%%%%%%%%%%%%%%%
%%%%%%%%%%%%%%%%%%%%%%%%%%%%%%%%%%%%
%%%%%%%%%%%%%%%%%%%%%%%%%%%%%%%%%%%%

%: COMANDOS PERSONALIZADOS
%%%%%%%%%%%%%%%%%%%%%%%%%%%%%%%%%%%%
%%%%%%%%%%%%%%%%%%%%%%%%%%%%%%%%%%%%
%%%%%%%%%%%%%%%%%%%%%%%%%%%%%%%%%%%%
%%%%%%%%%%%%%%%%%%%%%%%%%%%%%%%%%%%%
%%%%%%%%%%%%%%%%%%%%%%%%%%%%%%%%%%%%
%%%%%%%%%%%%%%%%%%%%%%%%%%%%%%%%%%%%

%: Redefinición de estilo de secciones
\titleformat{\section}
  {\normalfont\LARGE\bfseries}
  {\thesection.}{1em}{}

%: Redefinición de estilo de subsecciones
\titleformat{\subsection}
  {\normalfont\large}
  {\thesubsection}{1em}{}

%: Colores
% Comandos para resaltar con colores
\newcommand{\contrastcolor}{Blue3}
\newcommand{\headingscolor}{Blue1}
% Texto en color
\newcommand{\red}[1]{\textcolor{red}{\hspace{0pt}#1}}
\newcommand{\green}[1]{\textcolor{Green3}{\hspace{0pt}#1}}
\newcommand{\blue}[1]{\textcolor{SteelBlue3}{\hspace{0pt}#1}}
\newcommand{\garnet}[1]{\textcolor{DarkRed}{\hspace{0pt}#1}}
% En español
\let\rojo\red
\let\verde\green
\let\azul\blue
\let\granate\garnet

%: Estilo teoremas, lemas, definiciones...
\newtheorem{teorema}{Teorema}[section] % Teorema
\newtheorem{corolario}[teorema]{Corolario} % Corolario
\newtheorem{lema}[teorema]{Lema} % Lema
\newtheorem{proposicion}[teorema]{Proposición} % Proposición
\newtheorem{definicion}{Definición}[section] % Definición

\theoremstyle{definition}
\newtheorem*{obs}{Observación} % Observación

\theoremstyle{definition}
\newtheorem*{ej}{Ejemplo}
\def\eqref#1{\textcolor{DarkRed}{{(\ref{#1})}}} % Redefine el estilo de las referencias a ecuaciones

%%%%%%%%%%%%%%%%%%%%%%%%%%%%%%%%%%%%
%%%%%%%%%%%%%%%%%%%%%%%%%%%%%%%%%%%%
%%%%%%%%%%%%%%%%%%%%%%%%%%%%%%%%%%%%
%%%%%%%%%%%%%%%%%%%%%%%%%%%%%%%%%%%%
%%%%%%%%%%%%%%%%%%%%%%%%%%%%%%%%%%%%
%%%%%%%%%%%%%%%%%%%%%%%%%%%%%%%%%%%%


%: MACROS PERSONALIZADAS 
%%%%%%%%%%%%%%%%%%%%%%%%%%%%%%%%%%%%
%%%%%%%%%%%%%%%%%%%%%%%%%%%%%%%%%%%%
%%%%%%%%%%%%%%%%%%%%%%%%%%%%%%%%%%%%
%%%%%%%%%%%%%%%%%%%%%%%%%%%%%%%%%%%%
%%%%%%%%%%%%%%%%%%%%%%%%%%%%%%%%%%%%
%%%%%%%%%%%%%%%%%%%%%%%%%%%%%%%%%%%%

%: ABREVIATURAS BÁSICAS
%%%%%%%%%%%%%%%%%%%%%%%%%%%%%%%%%%%%
%%%%%%%%%%%%%%%%%%%%%%%%%%%%%%%%%%%%
%%%%%%%%%%%%%%%%%%%%%%%%%%%%%%%%%%%%
%%%%%%%%%%%%%%%%%%%%%%%%%%%%%%%%%%%%
%%%%%%%%%%%%%%%%%%%%%%%%%%%%%%%%%%%%

%: Conjuntos y estructuras
\newcommand{\N}{\mathbb{N}} % Naturales
\newcommand{\Z}{\mathbb{Z}} % Enteros
\newcommand{\Q}{\mathbb{Q}} % Racionales
\newcommand{\R}{\mathbb{R}} % Reales
\newcommand{\C}{\mathbb{C}} % Complejos
\renewcommand{\O}{\mathcal{O}} % Anillo de funciones o de enteros
\newcommand{\F}{\mathscr{F}} % Anillo de series formales
\newcommand{\poly}{\mathbb{P}} % Anillo de polinomios
\newcommand{\field}{\mathbb{K}} % Cuerpo genérico
\newcommand{\formal}{\field \llbracket T \rrbracket} % Anillo de series formales
\newcommand{\maximal}{\mathfrak{m}}

%: Operadores
% \ker ya viene definido
\DeclareMathOperator*{\coker}{coker} % Coker
% \dim ya viene definido
\DeclareMathOperator*{\codim}{codim} % Codimensión
\DeclareMathOperator*{\rg}{rg} % Rango
\DeclareMathOperator{\gr}{gr} % Grado
\DeclareMathOperator{\mcd}{mcd} % Máximo común divisor
\DeclareMathOperator{\mcm}{mcm} % Mínimo común múltiplo
\DeclareMathOperator{\sop}{sop} % Soporte
\DeclareMathOperator*{\Hom}{Hom} % Morfismos
\DeclareMathOperator*{\End}{End} % Endomorfismos
\DeclareMathOperator*{\Aut}{Aut} % Automorfismos
\DeclareMathOperator*{\spec}{Spec} % Espectro
\DeclareMathOperator*{\ord}{ord} % Orden
\DeclareMathOperator*{\res}{res} % Residuo
\DeclareMathOperator{\Arg}{Arg} % Argumento
\DeclareMathOperator{\Log}{Log} % Logaritmo principal
\DeclareMathOperator{\dlog}{dlog} % Derivada logarítmica
\renewcommand\Re{\operatorname{Re}} % Parte real
\renewcommand\Im{\operatorname{Im}} % Parte imaginaria

%: Lógica
\newcommand{\ssi}{\Leftrightarrow} % Si y solo si
\providecommand{\implied}{\impliedby} % Implica, implicado por

\providecommand{\contradiction}{\ensuremath{\mathbin{\color{\proofcolor}\Rightarrow\Leftarrow}}}
\let\contradiccion\contradiction % Símbolo contradicción

\newlength{\eqheight}
\settoheight{\eqheight}{$=$}
\newcommand{\eqdef}{\mathrel{\raisebox{0pt}[\eqheight][0pt]{$\overset{\text{\tiny\upshape def}}{=}$}}} % Símbolo para definir

\renewcommand{\qedsymbol}{$\blacksquare$} % Símbolo q.e.d

%: ENCABEZADO
\pagestyle{fancy}
\fancyhead[R]{\bfseries \textsc{\rightmark}}
\fancyhead[L]{\bfseries \textsc{\leftmark}}
\renewcommand{\chaptermark}[1]{\markboth{#1}{}}
\renewcommand{\sectionmark}[1]{\markright{#1}}

\let\oldheadrule\headrule
\renewcommand{\headrule}{\color{DarkRed}\oldheadrule} % Color del encabezado

%: FIN DE MACROS
%%%%%%%%%%%%%%%%%%%%%%%%%%%%%%%%%%%%
%%%%%%%%%%%%%%%%%%%%%%%%%%%%%%%%%%%%
%%%%%%%%%%%%%%%%%%%%%%%%%%%%%%%%%%%%
%%%%%%%%%%%%%%%%%%%%%%%%%%%%%%%%%%%%
%%%%%%%%%%%%%%%%%%%%%%%%%%%%%%%%%%%%
%%%%%%%%%%%%%%%%%%%%%%%%%%%%%%%%%%%%

\makeindex

\begin{document}
\thispagestyle{empty}

\definecolor{granate}{RGB}{79, 3, 3}
\pagecolor{granate}

%: Portada
\begin{center}

\includegraphics[scale=0.15]{usal logo.png}

\vspace{1cm}

\color{white}
{\large \textbf{Facultad de Ciencias}}\\[0.2cm]
{\large \textsc{Grado en Matemáticas}}

\vspace{2cm}

--------------------------------------------------------------------------------------------------------
\vspace{0.5cm}

{\LARGE \textbf{INTRODUCCIÓN AL CÁLCULO UMBRAL }}\\[0.3cm]


\vspace{0.5cm}

{\large \textsc{Trabajo Fin de Grado}}

\vspace{0.5cm}

--------------------------------------------------------------------------------------------------------

\vspace{1.5cm}

    {\textsc{Autor:}} \\[0.25cm] \textsc{\large{Lucas Cuerva Cuenca }} \\ [1cm] {\textsc{Tutor:}} \\[0.25cm] \textsc{\large{Luis Manuel Navas Vicente}} 

\vfill

\LARGE{Curso 2025/2026}

\end{center}
%: Fin de portada

%: Página vacía
\newpage
\nopagecolor
\color{black}
\hfill
\thispagestyle{empty} % Eliminar numeración de la página

%: Página con cita
\newpage
\thispagestyle{empty} % Eliminar numeración de la página
\begin{flushright}
    \vspace*{9cm}
    \small\textit{¡Oh santas matemáticas, ojalá pudieraris, \\con vuestro perpetuo trato, consolar el resto de mis días \\de la maldad del hombre y de la injusticia del Gran Todo!}
    \\[1em]\textsc{Conde de Lautremont}
\end{flushright}

%: Página vacía
\newpage
\hfill
\thispagestyle{empty} % Eliminar numeración de la página

%: Índice
\newpage
\renewcommand{\contentsname}{\LARGE Índice de contenidos}
{
    \hypersetup{linkcolor = .}
    \tableofcontents 
}

\thispagestyle{empty} % Eliminar numeración del índice

%: Página vacía
\newpage
\hfill
\thispagestyle{empty} % Eliminar numeración de la página

\setcounter{page}{0} % La enumeración comienza con la introducción
sudo \chapter*{Introducción}
\addcontentsline{toc}{chapter}{Introducción} % Para añadirla al índice a pesar de que no esté numerada
\markboth{INTRODUCCIÓN}{} % Para arreglar el encabezado

\lettrine{L}{a historia y} la psicología ya han probado que el ser humano, especie extremadamente sensible al cambio, es incapaz de aceptar la genialidad cuando esta se le presenta y prefiere, en contrapartida, adoptar la cómoda postura de rechazarla. Por ello, no son pocas las veces que, en su origen, una nueva área dentro de las matemáticas ha sido menospreciada y condenada al escepticismo, por una notable cantidad de personas, debido a una escasa solidez en su formalización, a pesar de su más que innegable utilidad. Así ocurrió, por ejemplo, con la teoría de conjuntos, hoy en día base de toda teoría matemática; el cálculo infinitesimal, base de una enorme parte de la física moderna; o la geometría no euclidiana, también de enorme utilidad dentro de la física para formular la Teoría de la Relatividad. Lo que hoy se conoce como \textit{cálculo umbral} (o también \textit{cálculo simbólico}) es solo uno más de estos ejemplos.

Esta rama tiene sus orígenes más primitivos entre mediados y finales del siglo XIX. A pesar de que ciertas fuentes se lo atribuyan a Édouard Lucas o a James Silvester (este último, el inventor del término \textit{cálculo umbral}), lo cierto es que John Blissard fue el primero en explorar las ideas fundacionales de esta disciplina. Blissard publica el primero de varios artículos en 1861 \cite{blissard}, mientras que los trabajos de Lucas se publican 15 años después. En justo en este artículo donde Blissard introduce la noción de \textit{notación representativa} para referirse a unas cantidades, mediante lo cual acaba deduciendo ciertas igualdades. Por desgracia, todos estos trabajos escaseaban una buena fundación teórica puesto que ciertos aspectos permanecían sin dilucidarse, lo que los dotaba de un aspecto casi mágico que desde luego no favoreció su acogida. No sería hasta que Rota y Roman publicaran una serie de artículos (quizás el más importante \cite{roman2}), medio siglo después, que se daría una presentación verdaderamente rigurosa de estas teorías, reviviendo de esta forma las ideas que Blissard, Lucas y Sylvester, entre otros, tuvieron en un comienzo. Posteriormente, ya en 1984, Roman escribiría \cite{roman1}, el cual se considera un clásico de la materia y la base de todo el cálculo umbral actual. 

En sus orígenes, la idea fundamental sobre la que giraba todo el cálculo umbral era la de ``intercambiar`` subíndices con exponentes, para obtener así, de una forma relativamente sencilla y mediante manipulaciones algebraicas, una serie de relaciones de suma utilidad. Para ver con una mayor claridad a lo que nos referimos con esto, conviene que nos centremos en el ejemplo más clásico dentro de la materia: los \textit{números de Bernoulli}.

Recordamos que estos números, $B_n$, se definen mediante el desarrollo en serie de Taylor de la función generadora:
\begin{equation*}
    \frac{x}{e^x-1} = \sum_{n=0}^\infty B_n\frac{x^n}{n!}
\end{equation*}
Una vez dada su definición, es habitual comenzar a deducir propiedades que los números de Bernoulli deben satisfacer. Una de la más fundamentales, y la única que debemos saber para lo que prosigue, es que verifican la ecuación genérica:

\begin{equation}\label{generic}
    B_0 = 1 \hspace{1cm} \sum_{r=0}^{n-1}\binom{n}{r}B_r = 0 \hspace{0.25cm} (n>1)
\end{equation} 
No es complicado advertir aquí que esta relación se asemeja sobremanera al teorema binomial usual, con la única salvedad de que en este caso contamos con una cantidad $B_r$ y no $B^r$, sea lo que fuera esa cantidad $B$. Es justo aquí donde el cálculo umbral entra en juego. Para poder hacer este cambio, se considera lo que se conoce como una ''cantidad representativa`` $B$ tal que $B_n$ sea ``equivalente`` a $B^n$, lo cual poría expresarse en términos simbólicos como $B_n \equiv B^n$ (aquí, el subíndice $n$ se piensa como la sombra, en latín ``umbra``, de donde proviene el nombre). Notése que las comillas aquí son importantes; este proceso carece de todo el rigor necesario e incluso estamos hablando de una supuesta relación de equivalencia que no hemos definido correctamente. 

En cualquier caso, si nos dejamos guiar por esta intución y hacemos la vista larga a la rigurosidad podríamos reescribir esta relación de la forma siguiente:
\begin{equation*}
    \sum_{r=0}^{n-1}\binom{n}{r}B_r \equiv \sum_{r=0}^{n-1}\binom{n}{r}B^r\equiv (B+1)^n-B^n = 0 \hspace{0.5cm} (n>1)
\end{equation*}

O lo que es lo mismo:
\begin{equation*}
    (B+1)^n = B^n \hspace{0.5em} (n>1)
\end{equation*}

No entraremos en muchos detalles pero, a la hora de la verdad, la cantidad representativa $B$ es lo que se conoce como un \textit{umbrae} y la equivalencia antes citada es la \textit{equivalencia umbral}.\footnote{El ``intercambio`` de exponentes con subíndices se corresponde entonces con lo que se denomina una \textit{evaluación} definida en un cierto anillo $\mathcal{D}[\mathcal{A}]$. Si se desea más información, consultar el desarrollo en \cite{rota}}. Para lo que aquí nos interesa, conviene saber que no todas las manipulaciones algebraicas habituales dan resultados correctos y se debe tener cierto cuidado a la hora de operar con los \textit{umbrae}. Ahora bien, esto no impide que, respetando ciertas reglas de las que aquí tampoco hablaremos, esta metodología no propicie resultados tan útiles como ciertos. Veámoslo, pues, para con los números de Bernoulli.

En primer lugar, por la serie de Taylor de la exponencial se llega a que:
\begin{equation*}
e^{Bx} \equiv \sum_{n=0}^\infty B^n\frac{x^n}{n!} \equiv \sum_{n=0}^\infty B_n\frac{x^n}{n!} \implies e^{Bx} \equiv \frac{x}{e^x-1}
\end{equation*}
Haciendo uso de esta relación y de las propiedades de la exponencial:
\begin{equation*}
e^{(B+1)x} - e^{Bx} \equiv x
\end{equation*}
de donde se recupera la ecuación \eqref{generic}.
Sustituyendo $x$ por $-x$ en la relación anterior se obtiene:
\begin{equation*}
e^{-Bx} \equiv \frac{-x}{e^{-x}-1} = \frac{xe^x}{e^x-1} \equiv e^{(B+1)x}
\end{equation*}
e igualando coeficientes:
\begin{equation*}
(B+1)^n \equiv (-B)^n
\end{equation*}
la cual, utilizando de nuevo la ecuación \eqref{generic} implica que:
\begin{equation*}
B_1 = -\frac{1}{2} \hspace{1cm} B_3 = B_5 = \dots = B_{2n+1} = 0 \hspace{0.2cm} (n>1)
\end{equation*}
Es más, este procedimiento nos permite obtener propiedades para los, más generales, \textit{polinomios de Bernoulli}. Como es bien sabido, estos se definen con la función generadora:
\begin{equation*}
    \frac{te^{xt}}{e^t-1} = \sum_{n\geq0}\frac{B_n(x)}{n!}t^n
\end{equation*}
Por ejemplo, rápidamente vemos que el miembro de la izquierda es equivalente a $e^{(B+x)t}$ y por tanto:
\begin{equation*}
    B_n(x) \equiv (B+x)^n \equiv \sum_{k=0}^n \binom{n}{k}B_{k}x^{n-k}
\end{equation*}
De aquí deducimos además:
\begin{equation*}
    B_n(0) \equiv B^n \equiv B_n \quad B_n'(x) \equiv n(B+x)^{n-1} \equiv nB_{n-1}(x)
\end{equation*}
y si seguimos este proceso podríamos obtener igualdades que llegan incluso hasta la conocida fórmula de sumación de Euler-McLaurin.
\bigbreak

Como se ha podido ver, esta es una metodología poderosa que  puede ser aprovechada por muchas de las conocidas sucesiones de polinomios: los propios polinomios de Bernoulli, de Hermite, de Euler o de Laguerre son solo unos pocos ejemplos. Así pues, como poco, el cálculo umbral podría sernos útil para la resolución de problemas orientados a diversas áreas de las matemáticas, dentro de los cuales estos polinomios aparezcan. No obstante, la realidad es que es mucho más versátil. En la actualidad, se conocen aplicaciones del cálculo umbral tan diversas que van desde la combinatoria, pasando por la probabilidad, la estadística, la teoría de grafos o la topología algebraica, hasta incluso la física. \footnote{Todas estas pueden consultarse en \cite{bucchianico}}

En la práctica, el cálculo umbral clásico se centra en el estudio específico de las sucesiones de Sheffer, empleando para ello la teoría de operadores y funcionales lineales sobre el espacio de polinomios junto a las series formales, es decir, no se preocupa por la convergencia de series o sucesiones. Esta es la forma en la que Roman lo hace en \cite{roman2} y la que se seguirá en el presente trabajo. La intención del mismo es exponer esta teoría traduciéndola a un lenguaje algebraico que nos resulte más familiar y dando un enfoque que combine, pues, el análisis junto con el álgebra. En este sentido, el trabajo se puede dividir en dos partes principalmente: 

\begin{itemize}
    \item En la primera de ellas se desarrolla toda la teoría del cálculo umbral.
    \item En la segunda se dan aplicaciones varias de esta teoría. Aquí se da la forma de obtener las principales propiedades de los polinomios de Bernoulli mediante técnicas umbrales, aclarando así lo expuesto antes, así como otras tantas sucesiones de polinomios conocidas.
\end{itemize}

%: Página vacía
\newpage
\hfill
\thispagestyle{empty} % Eliminar numeración de la página

\chapter{La conexión umbral}

Antes de comenzar, hemos tanto como aclarar parte de la notación como fijar las convenciones que utilizaremos a lo largo del trabajo. En lo que sigue:
\begin{itemize}
    \item $\field$ será un cuerpo de característica cero cualquiera (podemos suponer que es $\C$) y este será el cuerpo base donde trabajaremos. \footnote{También se han dado generalizaciones de la teoría clásica del cálculo umbral cuando $\field$ no es de característica cero. Un ejemplo de esto es \cite{verdoodt}}
    \item $\F = \formal$ será el anillo de series formales en la variable formal $T$. Además, supondremos conocidas tanto la teoría básica como los principales resultados que a esta conciernen, aunque damos el apéndice \ref{series formales} donde se puede consultar parte de esta información.
    \item Utilizaremos $\poly$ para referirnos a la $\field$-álgebra de polinomios (con la estructura de álgebra usual) en una variable, es decir, $\field[x]$.
    \item $\gr(p(x))$ simbolizará el grado del polinomio $p(x)$ y $p^`(x)$ simbolizará la derivada, en el sentido usual, del polinomio $p(x)$. En el caso de las derivadas sucesivas, estas se escribirán como $p^{(k)}(x)$.
    \item $\delta_{n,k}$ será la delta de Dirac, es decir:
    \begin{equation*}
        \delta_{n,k} (x) = \begin{dcases}
            1 & k=n\\
            0 & k \neq n
        \end{dcases}
    \end{equation*}
    \item $(x)_n$ simboliza al factorial descendente. Esto es:
    \begin{equation*}
        (x)_n = x(x-1) \dots (x-n+1)
    \end{equation*}
\end{itemize}

%: EL ÁLGEBRA UMBRAL
\section{El álgebra umbral}
\begin{definicion}
    Dado un espacio vectorial $V$, llamaremos \textbf{funcional lineal} a todo elemento $L \in V^*$, donde $V^*$ es el espacio dual de $V$.
\end{definicion}

Tal y como indica el título de la sección, nos centraremos en el estudio de los funcionales lineales sobre el espacio de polinomios $\poly$; es decir, las aplicaciones $L$:
\begin{align*}
    L &\colon \poly \longrightarrow \field\\
     & p(x) \mapsto \langle L | p(x) \rangle
\end{align*}
tales que:
\begin{equation*}
    \langle L | p(x) + \lambda q(x) \rangle = \langle L | p(x) \rangle + \lambda \langle  L | q(x) \rangle
\end{equation*}

Nótese que hemos fijado la notación $\langle L | p(x) \rangle$ para simbolizar la acción de $L$ sobre $p(x)$, típica de la física. Más tarde se aclarará el motivo de esta elección. 

\begin{obs}
    \begin{itemize}
        \item Por ahorro del lenguaje y sin que cause confusión, llamaremos a los funcionales lineales de $\poly$ simplemente \textit{\textbf{funcionales}}.
        \item El espacio vectorial de funcionales posee también estructura de $\field$-álgebra de forma natural, con el producto de funcionales
        \begin{align*}
            \mathbf{\cdot} \colon & \poly^* \times \poly^* \longrightarrow \poly^*\\
            & (L, M) \mapsto L \cdot M
        \end{align*}
        dado por $\langle L \cdot M | p(x) \rangle = \langle L | p(x) \rangle \cdot \langle M | p(x) \rangle$
    \end{itemize}
\end{obs}

El primer resultado importante que usaremos continuamente proviene del álgebra lineal.
\begin{teorema}
    Si $\{p_n(x)\}_{n\geq0}$ es una sucesión de polinomios tal que $\gr(p_n) = n$ para todo $n$, entonces dos funcionales $L$ y $M$ son iguales si y solo si $\langle L | p_n(x) \rangle = \langle  M | p_n(x) \rangle$ para todo $n$. En particular, $p_n(x) = x^n$ cumple el teorema.
\end{teorema}
\begin{demo}
    Basta con ver que los polinomios $p_n(x)$ así definidos forman una base de Hamel de $\poly$. Que son generadores está claro por ser cada polinomio $p_n(x)$ de grado $n$. Veamos que cada subconjunto finito de estos son linealmente independientes. Sean pues $p_{n_1}(x), \cdots , p_{n_r}(x)$ un subconjunto finito de $\{p_n\}_{n\geq0}$, que, sin pérdida de generalidad, podemos suponer ordenado con $\gr(p_{n_1}(x)) < \dots < \gr(p_{n_r}(x))$, y $\lambda_{n_1}, \dots , \lambda_{n_r} \in \field$ escalares tales que:
    \begin{equation*}
        \lambda_{n_1}p_{n_1}(x) + \cdots + \lambda_{n_r}p_{n_r}(x) = 0
    \end{equation*}
    Si escribimos cada polinomio $p_{n_l}(x)$ como $p_{n_l}(x) = \sum_{k=0}^{n_l}a_{k,n_l}x^k$, igualando coeficientes y teniendo en cuenta que $\gr(p_{n_1}(x)) < \dots < \gr(p_{n_r}(x))$ y que $a_{n_k,n_k} \neq 0$ para todo $k$, obtenemos las ecuaciones:
    \begin{align*}
        \lambda_{n_r}a_{n_r,n_r} = 0 &\implies \lambda_{n_r} = 0\\
        \lambda_{n_{r-1}}a_{n_{r-1},n_{r-1}} + \lambda_{n_r}a_{n_{r-1},n_r} = 0 &\implies \lambda_{n_{r-1}} = 0\\
        \dots\\
        \sum_{k=0}^r\lambda_{n_k}a_{0,n_k} = 0 &\implies \lambda_{n_1} = 0
    \end{align*}
    Luego, son linealmente independientes y se concluye.
\end{demo}
\begin{corolario}\label{spanning argument}
    Dada $\{p_n(x)\}_{n\geq0}$ sucesión de polinomios tal que $\gr(p_n(x)) = n$ para todo $n$, un funcional está completamente determinado por las constantes $\langle L | p_n(x) \rangle$. En particular:
    \begin{enumerate}
        \item Un funcional $L$ está completamente determinado por las constantes $\langle L | x^n \rangle$
        \item Si $\langle L | p_n(x) \rangle = 0$ para todo $n$, entonces $L = 0$.
    \end{enumerate}
\end{corolario}
Este teorema nos es de suma utilidad, ya que nos permite definir un funcional de una forma tan sencilla como dar tan solo su valor sobre los polinomios $x^n$. Lo aprovechamos ahora para obtener uno de los teoremas fundamentales del cálculo umbral, el cual nos permitirá interpretar cada serie formal como un funcional concreto.

Sea $T^k \in \F$ la serie formal con único término el factor $T^k$. En virtud de \ref{spanning argument}, esta serie nos permite definir el funcional $\omega_{T^k}$ que viene dado por:
\begin{equation}\label{omegaT^k}
    \langle \omega_{T^k} | x^n \rangle = n! \delta_{n,k}
\end{equation}
Es más que natural extender esta definición usando la linealidad a una serie formal genérica. Así pues, si $F(T) \in \F$ es una serie formal cualquiera, esta definirá el funcional $\omega_F$ que viene dado por:
\begin{equation} \label{omegaF}
    \langle \omega_F | x^n \rangle = n![F]_n
\end{equation}
Recíprocamente, si consideramos un funcional $L \in \poly^*$, este define la serie formal:
\begin{equation}\label{F_L}
    F_L(T) = \sum_{n=0}^\infty \frac{ \langle L | x^n \rangle}{n!} T^n 
\end{equation}
Esta correspondencia entre funcionales y series formales es la que nos permite formular y demostrar el siguiente teorema, base de todo el cálculo umbral.

\begin{teorema} \textbf{(isomorfismo umbral)} La aplicación $\phi$ definida como
    \begin{align*}
        \phi \colon & \poly^* \xrightarrow{\sim} \F\\
        & L \mapsto F_L(T) = \sum_{n=0}^\infty \frac{ \langle L | x^n \rangle}{n!} T^n \\
        & \omega_F \mapsfrom F(T)
    \end{align*}
    donde el funcional $\omega_F$ viene dado por $\langle \omega_F | x^n \rangle = n! [F]_n$, es un isomorfismo lineal. A este morfismo $\phi$ lo llamaremos el \textbf{isomorfismo canónico umbral} (abreviadamente, i.c.u).
\end{teorema}
\begin{demo}
    \begin{itemize}
        \item \textit{Biyectividad:}
        \begin{equation*}
            F(T) \longrightarrow \omega_F \longrightarrow F_{\omega_F}(T) = F(T)
        \end{equation*}
        puesto que $\langle \omega_F | x^n \rangle = n! \cdot a_n$
        \begin{equation*}
            L \longrightarrow F_L(T) \longrightarrow \omega_{F_L} = L
        \end{equation*}
        ya que, por \ref{spanning argument}, $L = \omega_{F_L} \iff \langle L | x^n \rangle = \langle \omega_{F_L} | x^n \rangle$, lo cual es cierto.
        \item \textit{Linealidad:}
        \begin{align*}
            \phi (L+cM) & = F_{L+M} (T) = \sum_{n=0} ^\infty \frac{\langle L + cM | p(x) \rangle}{n!}T^n \\
            & = \sum_{n=0} ^\infty \frac{\langle L | p(x) \rangle}{n!}T^n + c\sum_{n=0} ^\infty \frac{\langle M | p(x) \rangle}{n!}T^n \\ &= F_L(T) + c\cdot F_M(T) = \phi (L) + c \cdot \phi (M)
        \end{align*}
    \end{itemize}
\end{demo}

Una vez probado que $\phi$ es un isomorfismo lineal entre el espacio de funcionales $\poly^*$ y el espacio de series formales $\F$, es natural que a uno le surja ahora la siguiente pregunta: ¿es, además, un isomorfismo de álgebras? La respuesta es que, con la estructura usual de $\field$-álgebra de $\poly^*$, no tiene por qué serlo. Es fácil comprobar esto, ya que:
\begin{equation*}
    \phi(LM) = F_{LM}(T) = \sum_{n\geq0} \frac{ \langle LM | x^n \rangle}{n!} T^n = \sum_{n\geq0} \frac{ \langle L | x^n \rangle \langle M | x^n \rangle}{n!} T^n
\end{equation*}
y es evidente que, en lo general:
\begin{equation*}
    \sum_{n\geq0} \frac{ \langle L | x^n \rangle \langle M | x^n \rangle}{n!} T^n \neq (\sum_{n\geq0} \frac{ \langle L | x^n \rangle}{n!} T^n) (\sum_{n\geq0} \frac{ \langle L | x^n \rangle}{n!} T^n) = \phi(L) \cdot \phi(M)
\end{equation*}
Ahora bien, también es cierto que este isomorfismo induce una estructura de $\field$-álgebra en $\poly^*$ distinta de la que en un principio teníamos y para la cual $\phi$ sí será un isomorfismo de álgebras de forma evidente. Esta es la que llamaremos \textit{álgebra umbral}.

\begin{definicion}
    Llamaremos \textbf{álgebra umbral} al espacio vectorial $\poly^*$ dotado de la estructura de $\field$-álgebra definida por:
    \begin{equation}\label{producto funcionales}
        \langle LM | x^n \rangle = \sum_{k=0}^n\binom{n}{k}\langle L | x^k \rangle \langle M | x^{n-k} \rangle
    \end{equation}
\end{definicion}
Es una mera comprobación ver que, en efecto, esto nos da una estructura de $\field$-álgebra y en forma de resumen de lo dicho hasta ahora, llegamos al siguiente resultado.

\begin{corolario}
    El i.c.u $\phi$ es un isomorfismo de álgebras entre el álgebra umbral y el álgebra de series formales.
\end{corolario}

En particular, el álgebra umbral cumple las siguientes dos propiedades.
\begin{proposicion}\label{identidad algebra umbral}
    El funcional $e^{0T} = 1$ es la identidad para el producto del álgebra umbral.
\end{proposicion}
\begin{demo}
    Basta con usar el apartado 1 de \ref{spanning argument}.
\end{demo}

\begin{proposicion}\label{producto multiples funcionales}
    Sean $L_1, \dots , L_r \in \poly^*$ funcionales. Entonces,
    \begin{equation*}
        \langle L_1 \cdots L_m | x^n \rangle = \sum_{|i|=n}\binom{n}{i_1, \dots , i_m} \langle L_1| x^{i_1} \rangle \dots  \langle L_m | x^{i_m} \rangle
    \end{equation*}
    donde $|i| = i_1 + \dots +i_m$ y $\binom{n}{i_1, \dots , i_m} = \frac{n!}{k_1! \cdots k_m!}$
\end{proposicion}
\begin{demo}
    Se ve aplicando inducción a \eqref{producto funcionales}
\end{demo}

\begin{obs}
    Este último resultado se puede generalizar aún más, pues es válido para todas las sucesiones de polinomios que son de \textit{\textbf{tipo binomial}}. Más tarde las definiremos correctamente y veremos que $p_n(x) = x^n$ es una de estas sucesiones. No obstante, nos es suficiente con conocerlo para el caso particular de $x^n$, tal y como se ha formulado en \ref{producto multiples funcionales}.
\end{obs}

El espacio de funcionales también puede tener una estructura de espacio topológico. Para ello, escogemos la topología siguiente, con la cual $\phi$ será continua.
\begin{definicion}
    Dados dos espacios vectoriales $V$ y $W$, se define la \textbf{topología de estabilización de} $\Hom(V,W)$ como aquella dada por la siguiente condición de convergencia: una sucesión de homomorfismos $\{f_n\}_n \subset \Hom(V,W)$ converge a un homomorfismo $f$ si, dado un elemento $v \in V$, existe un número natural $n_0$ tal que si $n \geq n_0$ entonces $f_n(v) = f(v)$, es decir, cuando $\{f_n(v)\}_{n}$ converge en la topología discreta de $W$.
\end{definicion}

\begin{proposicion}
    Con la topología de estabilización y la estructura de álgebra umbral, el espacio de funcionales $\poly^*$ es un álgebra topológica completa.
\end{proposicion}
\begin{demo}
    Ambos hechos son consecuencias inmediatas de la definición dada de convergencia para funcionales. Para ver que es un álgebra topológica basta con probar que tanto la suma como el producto de operadores son aplicaciones continuas, ya que, por la definición del producto de operadores, esto implicaría también que el producto por escalares es continuo. 
    
    Consideramos entonces sucesiones de operadores $L_n$ y $M_n$ que convergan a ciertos operadores $L$ y $M$, respectivamente; es decir, ambas sucesiones se estabilizan para un $n$ lo suficientemente grande. Por tanto, tendremos que
    \begin{align*}
        \langle L_n + M_n | x^m \rangle & = \langle L_n | x^m \rangle + \langle M_n | x^m \rangle  = \langle L | x^m \rangle + \langle M | x^m \rangle = \langle L + M | x^m \rangle  \\
        \langle L_n \cdot M_n | x^m \rangle & = \sum_{k=0}^m \binom{m}{k}\langle L_n | x^k \rangle \langle M_n | x^{m-k} \rangle = \sum_{k=0}^m \binom{m}{k}\langle L | x^k \rangle \langle M | x^{m-k} \rangle = \langle L \cdot M | x^m \rangle
    \end{align*}
    para un $n$ suficientemente grande y para todo $m \geq 0$. Generalizando esto por linealidad a un polinomio $p(x) \in \poly$ cualquiera, obtenemos que $L_n + M_n$ converge a $L + M$ y $L_n \cdot M_n$ converge a $L \cdot M$, con lo que se concluye.

    Queda ver que es completa. Sea $\{L_n\}_{n}$ una sucesión de Cauchy en $\poly^*$; es decir, dado polinomio $p(x) \in \poly^*$ existe un $n_0$ tal que si $n, m \geq n_0$ entonces $\langle L_n | p(x) \rangle = \langle L_m | p(x) \rangle $. En particular, para el operador $L$ definido por $\langle L | p(x) \rangle \equiv \langle L_{n_0} | p(x) \rangle $, tenemos que
    \begin{equation*}
       \langle L_n | p(x) \rangle = \langle L_{n_0} | p(x) \rangle = \langle L | p(x) \rangle
    \end{equation*}
    Luego, $L_n \longrightarrow L$ y queda así probado que el espacio es completo.
\end{demo}

\begin{proposicion}
    Con la topología de estabilización de $\End(\poly)$ y la topología formal, el i.c.u $\phi$ es un homeomorfismo.
\end{proposicion}

\begin{demo}
    \begin{itemize}
        \item \textit{$\phi$ es continua:} Sea $\{L_n\}_{n}$ sucesión de funcionales tal que converja a un funcional $L$. Esto es, dado $p(x) \in \poly$ existe un $n_0$ tal que si $n \geq n_0$ entonces $\langle L_n | p(x) \rangle = \langle L | p(x) \rangle$. Por otro lado, tenemos que
        \begin{equation*}
            \phi(L_n) = F_{L_n}(T) = \sum_{k\geq0} \frac{\langle L_n | x^k \rangle}{k!}T^k \quad \phi(L) = F_{L}(T) = \sum_{k\geq0} \frac{\langle L | x^k \rangle}{k!}T^k
        \end{equation*}
        Luego, $[F_{L_n}]_k = [F_L]_k$ para un $n$ lo suficientemente grande y para todo $k \geq 0$. Esto no es más que la convergencia de $F_{L_n}$ a $F_L$ en la topología formal y, por tanto, $\phi$ es continua.
        \item \textit{$\phi^{-1}$ es continua:} De forma similar, sea $\{F_n(T)\}_{n}$ sucesión de series formales tal que converja a una serie formal $F(T)$, siendo
    \begin{equation*}
        F_n(T) = \sum_{k\geq 0} \frac{a_{n, k}}{k!}T^K \quad F(T) = \sum_{k\geq 0} \frac{a_{k}}{k!}T^K 
    \end{equation*}
    Por tanto, existe un natural $n_0$ tal que $a_{n,k} = a_k$ siempre que $n \geq n_0$. Luego, por la definición de la aplicación $\phi^{-1}$, obtenemos que
    \begin{equation*}
        \langle \phi^{-1}(F_n(T)) | x^m \rangle = \langle \omega_{F_n} | x^m \rangle = m!a_{n,m} = m! a_m = \langle \phi^{-1}(F(T)) | x^m \rangle
    \end{equation*}
    para $n \geq n_0$ y para todo $m \geq 0$. Luego, por la linealidad, se ve que $\langle \phi^{-1}(F_n(T)) | p(x) \rangle = \langle \phi^{-1}(F(T)) | p(x) \rangle$ para todo polinomio $p(x)$ y para un $n$ lo suficientemente grande. Con lo que $\phi^{-1}(F_n(T)) \longrightarrow \phi^{-1}(F(T))$ y, por tanto, $\phi^{-1}$ es continua.
    \end{itemize}
\end{demo}

\begin{proposicion}
    La topología de estabilización de $\End(\poly)$ coincide con la topología inicial del i.c.u $\phi$.
\end{proposicion}

Por otro lado, de la definición \ref{omegaF} dada es fácil ver la propiedad siguiente. 

\begin{proposicion}
    Sean $F(T), G(T) \in \F$ dos series formales. Entonces, $F(T) = G(T) $ si y solo si $\omega_F = \omega_G$.
\end{proposicion}

De esta forma, podemos pensar cada serie formal $F(T)$ como el funcional $\omega_F$ y escribir indistintamente $\F$ o $\poly^*$ para referirnos a ambos conjuntos. Haciendo abuso de la notación, escribiremos $F(T)$ para denotar tanto a la serie formal como al funcional asociado $\omega_F$; cómo esta se comporta dependerá del contexto en el que aparezca y no debe generar confusión alguna. Para cerrar la sección, damos algunos ejemplos básicos de funcionales que nos serán de enorme utilidad. 

\begin{ej}
\begin{enumerate}
    \item El \textbf{funcional derivada $k$-ésima} $T^k$ viene dado directamente por \ref{omegaT^k}:
    \begin{equation*}
        \langle T^k| x^n \rangle = n!\delta_{n,k} = \begin{dcases}
            n! & k=n\\
            0 & k \neq n
        \end{dcases}
        \quad =  (x^n)^{(k)}(0)
    \end{equation*}
    Y si tomamos un polinomio $p(x)$ cualquiera:
    \begin{equation}\label{funcional derivada}
        \langle T^k| p(x) \rangle = p^{(k)}(0)
    \end{equation}
    \item El \textbf{funcional evaluación} $e^{\lambda T}$ está dado por:
    \begin{equation*}
        \langle e^{ \lambda T} | x^n \rangle = \langle \sum_{n\geq0} \frac{\lambda^n}{n!}T^n | x^n \rangle = \lambda^n 
    \end{equation*}
    Y, por tanto:
    \begin{equation}\label{funcional evaluacion}
        \langle e^{\lambda T} | p(x) \rangle = p(\lambda)
    \end{equation}
    Simbolizaremos a este funcional por $\varepsilon_\lambda \equiv e^{\lambda T}$ y en caso de que $\lambda = 0$, se dice que $\varepsilon_0$ es el \textbf{funcional aumento}. Con esta notación, la proposición \ref{identidad algebra umbral} se reformula diciendo que el funcional aumento es la identidad para el producto del álgebra umbral.
    \item El \textbf{funcional diferencia} $e^{ \lambda T} -1$ está dado por:
    \begin{equation}\label{funcional diferencia}
        \langle e^{\lambda T}-1 | p(x) \rangle = p(\lambda)-p(0)
    \end{equation}
    \item El \textbf{funcional de Abel} $Te^{\lambda T}$ viene dado por:
    \begin{equation*}
        \langle Te^{\lambda T} | x^n \rangle = \langle \sum_{n\geq0} \frac{\lambda^n}{n!}T^{n+1} | x^n \rangle = nx^{n-1}
    \end{equation*}
    Con lo que:
    \begin{equation}\label{funcional de abel}
        \langle Te^{\lambda T} | p(x) \rangle = p^`(\lambda)
    \end{equation}
\end{enumerate}
\end{ej}


%%%%%%%%%%%%%%%%%
%%%%%%%%%%%%%%%%%
%%%%%%%%%%%%%%%%%
%%%%%%%%%%%%%%%%%
%%%%%%%%%%%%%%%%%

%: PROPIEDADES DE LOS FUNCIONALES
\section{Propiedades de los funcionales}
Mostramos aquí las propiedades fundamentales de los funcionales lineales sobre $\poly$, consecuencias todas ellas, más o menos directas, de su interpretación como series formales mediante el i.c.u., así como de la notación establecida.

A partir de las definiciones dadas y de las igualdades \ref{funcional derivada} y \ref{omegaF} se obtiene un análogo del teorema de Taylor clásico para las series formales (o funcionales), así como una reescritura del mismo en términos de los funcionales $T^n$.

\begin{proposicion}\label{taylor analogo}
    Sea $F(T) \in \F$ y $p(x) \in \poly$. Entonces:
    \begin{enumerate}
        \item $\begin{aligned}[t]
            F(T) = \sum_{n \geq 0} \frac{ \langle F(T) | x^n \rangle}{n!}T^n \quad \text{\textbf{(desarrollo de Taylor para funcionales)}}
        \end{aligned}$ 
        \item $\begin{aligned}
            p(x) = \sum_{n \geq 0} \frac{ \langle T^n| p(x) \rangle}{n!}x^n \quad \text{\textbf{(desarrollo de Taylor clásico)}}
        \end{aligned}$
    \end{enumerate}
\end{proposicion}

\begin{proposicion} \label{orden mayor que grado}
    Sean $F(T) \in \F$ y $p(x) \in \poly$. Si $\ord(F(T)) > \gr(p(x))$, entonces $\langle F(T)| p(x) \rangle = 0$
\end{proposicion}
\begin{demo}
    Esto es consecuencia de la definición \ref{omegaF}, pues $$\langle F(T)| x^n\rangle = n! [F]_n \implies \langle F(T)| p(x)\rangle = \sum_{n}n![F]_n[p]_n$$ y como $\ord(F(T)) > \gr(p(x))$, o bien $[F]_n = 0$ o bien $[p]_n = 0$.    
\end{demo}

\begin{proposicion}
    Sea $F(T) \in \F$, $p(x) \in \poly^*$ y $\lambda \in \field$ cualesquiera. Se verifica que
    \begin{equation*}
        \langle F(T)| p(\lambda x)\rangle = \langle F(\lambda T) | p(x)\rangle
    \end{equation*}
\end{proposicion}
\begin{demo}
    Para $(\lambda x)^k$ el resultado es cierto puesto que
    \begin{equation*}
        \langle T^n| (\lambda x)^k \rangle = \lambda^k n! \delta_{n,k} = \lambda^n n! \delta_{n,k} = \langle (\lambda T)^n| x^k\rangle
    \end{equation*}
    Para un polinomio genérico $p(x)$, por la proposición anterior y la linealidad, escribiendo $m = \gr(p(x))$ se verifica que
    \begin{equation*}
        \langle F(T) | p(\lambda x) \rangle = \sum_{k=0}^m \sum_{n=0}^m [p]_k[F]_k \langle T^n| (\lambda x)^k\rangle = \sum_{k=0}^m \sum_{n=0}^m [p]_k[F]_k \langle (\lambda T)^n| x^k\rangle = \langle F(\lambda T) | p(x)\rangle
    \end{equation*}
\end{demo}
\begin{proposicion}
    Sea $\{ F_k(T) \}_{k\geq0}$ sucesión de series formales tal que $\ord(F_k) \rightarrow \infty$ cuando $k\rightarrow \infty$. Entonces $\sum_{k}a_kF_k(T)$ converge en la topología formal y además se verifica que:
    \begin{equation*}
        \langle \sum_k a_kF_k(T)| p(x)\rangle =  \sum_k a_k\langle F_k(T)| p(x)\rangle
    \end{equation*}
    para todo $p(x) \in \poly$. En particular, se cumple cuando $\ord(F_k(T)) = k$
\end{proposicion}
\begin{demo}
Sea $m$ tal que $\ord(F_k) > \gr(p(x))$ para todo $k > m$. De nuevo, por la proposición \ref{orden mayor que grado} tenemos que
    \begin{align*}
        \langle \sum_{k=0}^\infty a_kF_k(T) | p(x) \rangle & = \langle \sum_{k \leq m} a_kF_k(T) + \sum_{k > m} a_kF_k(T)| p(x) \rangle \\& = \langle \sum_{k \leq m} a_kF_k(T)| p(x) \rangle = \sum_{k \leq m} a_k\langle F_k(T)| p(x)\rangle = \sum_{k=0}^\infty a_k\langle F_k(T)| p(x)\rangle
    \end{align*}
    donde la segunda suma es finita ya que $\langle F_k(x) | p(x) \rangle = 0$ para todo $k > m$.
\end{demo}

\begin{proposicion}
    Sea $\{ F_k(T) \}_{k\geq0}$ sucesión de series formales tal que $\ord(F_k(T)) = k$. Entonces la aplicación
    \begin{align*}
        \langle F(T) \rangle \colon & \poly \longrightarrow \field^\infty\\
        &p(x) \mapsto \{ \langle F_k(T) | p(x) \rangle \}_{k \geq 0}
    \end{align*}
    que asocia a cada polinomio la sucesión $\{ \langle F_k(T) | p(x) \rangle \}_{k \geq 0}$, es inyectiva.
\end{proposicion}
\begin{demo}
    Ya que $\ord(F_k(T)) = k$, sabemos que las series formales $F_k(T)$ forman una pseudobase de $\F$. En particular, existen $a_{n,k} \in \field$ tales que $T^n = \sum_{k} a_{n,k}F_k(T)$. Por tanto, por la proposición anterior:
    \begin{equation*}
        p^{(n)}(0) = \langle T^n | p(x) \rangle = \langle \sum_{k} a_{n,k}F_k(T) | p(x) \rangle = \sum_{k} a_{n,k} \langle F_k(T) | p(x) \rangle
    \end{equation*}
    Con lo que:
    \begin{equation*}
        \langle F(T) \rangle(p(x)) =\{  \langle F_k(T) | p(x) \rangle \}_{k \geq 0} = 0 \implies \sum_{k} a_{n,k} \langle F_k(T) | p(x) \rangle = 0 \implies p^{(n)}(0) =0 \quad \forall n 
    \end{equation*}
    Luego, $p(x) = 0$ necesariamente y se concluye.
\end{demo}

\begin{proposicion}
    Sea $\{p_k(x)\}_{k \geq 0}$ familia de polinomios que sea un sistema de generadores algebraico para $\poly$. Entonces, la aplicación:
    \begin{align*}
        \langle p(x) \rangle \colon & \poly^* \longrightarrow \field^\infty\\
        &F(T) \mapsto \{  \langle F(T) | p_k(x) \rangle \}_{k \geq 0}
    \end{align*}
    es inyectiva. En particular, el resultado se verifica para los $p_k(x)$ tales que $\gr(p_k(x)) = k$. 
\end{proposicion}
\begin{demo}
    Como la anterior, escribiendo $x^n$ como combinación lineal de los $p_k(x)$. 
\end{demo}

\begin{proposicion}\label{adjuncion M}
    Sean $F(T) \in \F$ y $p(x) \in \poly$ cualesquiera. Entonces, se verifica que
        $$\begin{aligned}
            \langle \partial F(T) | p(x) \rangle = \langle F(T) | x \cdot p(x)\rangle
        \end{aligned}$$
\end{proposicion}

\begin{demo}
    $F(T) = \sum_k a_kT^k$. Por la linealidad, basta verlo para $p(x) = x^n$. Tenemos que: 
        $$\begin{aligned}
            \langle \partial F(T)| x^n\rangle & = \sum_{k=1}^\infty ka_k\langle T^{k-1}| x^n\rangle = \sum_{k=0}^\infty (k+1)a_{k+1}\langle T^{k}| x^n\rangle \\
            & = \sum_{k=0}^\infty (k+1)a_{k+1}k! \delta_{n,k} = (n+1)!a_{n+1} = \langle F(T)| x^{n+1}\rangle \\
            &= \langle F(T)| x\cdot x^n\rangle
        \end{aligned}$$
\end{demo}
    
\begin{proposicion}
    Si $F(T) \in \F$ es una serie formal tal que $\ord(F(T)) \geq 1$, entonces:
    \begin{equation*}
        \langle \frac{F(T)}{T} | p(x) \rangle = \langle F(T) | \int p(x) \rangle
    \end{equation*}
\end{proposicion}

\begin{demo}
    Si $F(T) = \sum_{k\geq1} a_kT^k$ tenemos que
    \begin{align*}
        \langle \frac{F(T)}{T} | x^n \rangle & = \langle \sum_{k=1}^\infty a_kT^{k-1}| x^n \rangle = \sum_{k=0}^\infty a_{k+1} \langle T^{k}| x^n \rangle\\ 
        & = n! a_{n+1} = (n+1)!\frac{a_{n+1}}{(n+1)} = \langle F(T) | \int x^n \rangle
    \end{align*}
    Extendiéndolo a todo $p(x) \in \poly$ por linealidad se concluye.
\end{demo}
    
\begin{ej}
    El funcional $\frac{e^{\lambda T} -1}{T}$ (bien definido ya que $\ord(e^{\lambda T} - 1) = 1$) será aquel con
    \begin{equation*}
        \langle \frac{e^{\lambda T} -1}{T} | p(x) \rangle = \langle e^{\lambda T} -1 | \int p(x) \rangle = (\int p(x))(\lambda) - (\int p(x))(0) = (\int p(x))(\lambda)
    \end{equation*}
    En el caso de que $\field = \C, \R$ entonces
    \begin{equation} \label{funcional Bernoulli}
        \langle \frac{e^{\lambda T} -1}{T} | p(x) \rangle = \int_{0}^{\lambda} p(u), du
    \end{equation}
    Este funcional nos aparecerá cuando estudiemos los polinomios de Bernoulli y allí lo analizaremos con detenimiento. No obstante, de momento basta con observar que para $\lambda = 1$ el funcional inverso $\frac{T}{e^{T} -1}$ cumple que
    \begin{equation*}
        \frac{e^{T} -1}{T} = \sum_{k \geq 0} \frac{B_k}{k!}T^k \implies \frac{e^{\lambda T} -1}{T} = \lambda^{-1}\sum_{k \geq 0} \frac{B_k \lambda^k}{k!}T^k 
    \end{equation*}
    Luego, por \ref{taylor analogo}
    \begin{equation*}
        \langle \frac{T}{e^{\lambda T}-1} | x^n\rangle = \lambda^{-1} n! \frac{B_n}{n!} \lambda^n = B_n \lambda^{n-1}
    \end{equation*}
\end{ej}

Dos familias de funcionales son de especial importancia: estos son los \textit{funcionales delta} y los \textit{funcionales invertibles}. Estos nos aparecerán, fundamentalmente, cuando hablemos de las \textit{sucesiones de Sheffer}. Damos tanto sus definiciones como su caracterización.

\begin{definicion}
    Dada $F(T) \in \F$, diremos que $F(T)$ es una \textbf{serie delta} (o \textbf{$\delta$-serie}) si $F(T) \in \maximal$ y $F(T) \not \in \maximal^2$; es decir, cuando $F(T)$ es un invertible respecto de la composición. Escribimos $\F^\delta$ para simbolizar el conjunto de series delta.
\end{definicion}

\begin{definicion}
    Dada un funcional $L \in \poly^*$, diremos que $L$ es un \textbf{funcional delta} (o \textbf{$\delta$-funcional}) si al pensarlo como una serie (es decir, $\phi(L)$) es una serie delta. Análogamente, diremos que $L$ es un \textbf{funcional invertible} si al pensar $L$ como una serie formal esta es una serie invertible.
\end{definicion}

\begin{proposicion}
    Sea $F(T) \in \F$. Entonces:
    \begin{enumerate}
        \item $F(T) \in \F^{\delta} \iff \langle F(T)| 1\rangle = 0, \langle F(T)| x \rangle \neq 0$
        \item $F(T) \in \F^* \iff \langle F(T)|1 \rangle \neq 0$
    \end{enumerate}
\end{proposicion}

\begin{demo}
    Se sigue de las definiciones y de las propiedades para series formales.
\end{demo}

%%%%%%%%%%%%%%%%%%%%%
%%%%%%%%%%%%%%%%%%%%%
%%%%%%%%%%%%%%%%%%%%%
%%%%%%%%%%%%%%%%%%%%%
%%%%%%%%%%%%%%%%%%%%%

%: OPERADORES LINEALES
\section{La relación con los operadores lineales}

\begin{definicion}
    Sea $V$ un espacio vectorial cualquiera. Llamamos \textbf{operador lineal} de $V$ a todo elemento $\Lambda$ de $\End(V)$.
\end{definicion}

\begin{obs}
    De forma similar a lo que hacíamos con los funcionales, a los operadores lineales de $\End(\poly)$ los llamaremos simplemente \textbf{operadores}. Siguiendo esta línea, recordamos que el espacio de operadores, $\End(\poly)$,  tiene estructura de $\field$-álgebra, siendo el producto de operadores el producto usual.
\end{obs}



Previamente hemos identificado el espacio de series formales, $\F$, con el espacio de funcionales, $\poly^*$, y como consecuencia hemos dotado a $\poly^*$ de una estructura de álgebra, inducida por el álgebra de las series formales. Una nueva interpretación de las series formales nos permite ahora dotar a $\poly$ de una estructura de $\F$-módulo.

Para ello, consideramos la identificación
\begin{align} \label{T como derivada}
    & \F \longrightarrow \End(\poly) \nonumber \\
    & T^k \longleftrightarrow \partial^k
\end{align}
donde es importante recordar que $\partial^k$ actúa como:
\begin{align*}
    \partial^kx^n = \begin{dcases}
        (n)_kx^{n-k} & k \leq n\\
        0 & k > n
    \end{dcases}
\end{align*}
Ahora podemos extender esta asignación por linealidad a la aplicación $\varphi$ definida por
\begin{align*}
    \varphi \colon & \F \longrightarrow \End(\poly)\\
    & \sum_{k\geq0} \frac{a_k}{k!}T^k \mapsto \sum_{k \geq 0} \frac{a_k}{k!}\partial^k
\end{align*}
de tal forma que obtenemos el operador:
\begin{align*}
    F(\partial)x^n =\sum_{k \geq 0} \binom{n}{k}a_kx^{n-k}
\end{align*}
Hemos de hacer dos aclaraciones importantes respecto a lo dicho hasta ahora. Primero que todo, se puede observar que hemos escrito los elementos de $\F$ en notación exponencial (es decir, con un factor $\frac{1}{k!}$). Esto es así puesto que nos resultará más conveniente lidiar con los coeficientes binomiales que con los factoriales descendentes, y en lo que sigue esta será la convención que utilizaremos. Por otro lado, del razonamiento seguido se puede inferir también que la convención seguida para simbolizar la acción de un operador $\Lambda$ sobre un polinomio $p(x)$ es la de escribir la yuxtaposición del símbolo del operador con el del polinomio: $\Lambda p(x)$. Esto no debe causar confusión con el producto de operadores y, además, es el motivo principal por el que escogimos la notación $\langle F(T)| p(x)\rangle$ para la acción de los funcionales.

De forma similar a lo que hacíamos con los funcionales, utilizaremos el símbolo $T$ para hablar tanto de la variable formal como del operador $\partial$, con lo que escribiremos siempre $F(T)$ en lugar del operador asociado $\varphi(F(T))$. En caso de no indicarlo, simplemente el contexto ha de servir para diferenciar como se está considerando.

En resumen, dada $F(T)$ la escribiremos como $F(T) = \sum_k \frac{a_k}{k!}T^k$ y esta actúa sobre $\poly$ como
\begin{align}\label{serie como operador}
    F(T) \colon & \poly \longrightarrow \poly \nonumber \\
    & x^n \mapsto F(T)x^n = \sum_k \binom{n}{k}a_kx^{n-k}
\end{align}

Con todo, podemos ya enunciar y probar el resultado principal de la sección.

\begin{teorema}
    El producto definido por
    \begin{align*}
        \cdot \colon & \F \times \poly \longrightarrow \poly\\
        & (F(T), p(x)) \mapsto F(T) \cdot p(x) \equiv F(T) p(x)
    \end{align*}
    siendo $F(T) p(x)$ la acción definida por \eqref{serie como operador}, dota a $\poly$ de una estructura de $\F$-módulo. Además, es un $\F$-módulo fiel.
\end{teorema}
\begin{demo}
    A partir de la definición de $\cdot$ es claro que
    \begin{align*}
        (F(T) + G(T))\cdot p(x) & = F(T) \cdot p(x) + G(T) \cdot p(x)\\
        F(T)\cdot (p(x) + q(x)) & = F(T) \cdot p(x) + F(T) \cdot q(x)\\
        1 \cdot p(x) & = p(x)
    \end{align*}
    Para ver que $F(T) \cdot (G(T) \cdot p(x)) = (F(T)G(T))\cdot p(x)$ basta con tener en cuenta que
    \begin{align*}
        T^k T^j \cdot p(x) = T^{k+j} \cdot p(x) = \partial^{k+j} \cdot p(x) = \partial^k(\partial^j p(x)) = T^k \cdot (T^j \cdot p(x)) 
    \end{align*}
    y por la linealidad se concluye que es en efecto un $\F$-módulo.

    Queda probar que el módulo es fiel. Esto es que
    \begin{equation*}
        \text{Ann}(\poly) = \{ F(T) \in \F : F(T) \cdot p(x) = 0, \forall p(x) \in \poly\} = \{0\}
    \end{equation*}
    Sea, pues, $F(T) = \sum_k \frac{a_k}{k!}T^k \in \text{Ann}(\poly)$. Aplicándola a cada $x^n$ obtenemos que
    \begin{align*}
        F(T) \cdot x^0 & = a_0 = 0\\
        F(T) \cdot x^1 & = a_0x + a_1 = 0\\
        F(T) \cdot x^2 & = a_0x^2 + 2a_1x + a_2 = 0\\
        \vdots
    \end{align*}
    de donde vemos que $a_0 = a_1 = \dots = a_n = \dots = 0$ y se concluye.
\end{demo}

Los operadores lineales $F(T)$ cumplirán propiedades análogas a la de los funcionales. Listamos algunas de estas aunque no damos su demostración puesto que son muy similares a las dadas para estos útlimos.

\begin{proposicion}
    Si $\ord(F) > \gr(p)$, entonces $F(T)p(x) = 0$.
\end{proposicion}

\begin{proposicion}
    Si $\{F_k(T)\}_{k \geq 0}$ es una sucesión de series formales tal que $\ord(F_k) \rightarrow \infty$, entonces
    \begin{equation*}
        (\sum_{k=0}^\infty a_kF_k(T))p(x) = \sum_{k=0}^\infty a_k(F_k(T)p(x))
    \end{equation*}
    para todo polinomio $p(x) \in \poly$. En particular, un operador está completmaente determinado por su acción sobre los elementos de una pseudobase de $\F$.
\end{proposicion}


\begin{proposicion}
    Si $\{ F_k(T) \}_{k \geq 0}$ es un sistema de generadores topológico para $\F$, entonces 
    \begin{equation*}
        F_k(T)p(x) = 0, \forall k \iff p(x) = 0
    \end{equation*}
    para todo $p(x) \in \poly$
\end{proposicion}

\begin{proposicion}
    Si $\{ p_k(x) \}_{k \geq 0}$ es un sistema de generadores algebraico para $\poly$, entonces 
    \begin{equation*}
        F(T)p_k(x) = 0, \forall k \iff F(T) = 0
    \end{equation*}
    para toda $F(T) \in \F$
\end{proposicion}

Uno de los resultados claves en lo que a estos operadores respecta es el de la relación de adjunción que guardan con el producto del álgebra formal.
\begin{teorema} \label{adjuncion serie operador y multiplicacion}
    Sean $F(T), G(T) \in \F$. Para todo $p(x) \in \poly$ se verifica
    \begin{equation*}
        \langle F(T)G(T) | p(x) \rangle = \langle F(T) | G(T)p(x) \rangle
    \end{equation*}
\end{teorema}

\begin{demo}
    Por la linealidad, esto se reduce a ver que $\langle T^k T^j | p(x) \rangle = \langle T^k | T^j p(x) \rangle$. En efecto:
    \begin{align*}
        \langle T^k T^j | p(x) \rangle & = \langle T^{k+j} | p(x) \rangle = p^{(k+j)}(0)\\
        & = (\partial^j p)^{(k)}(0) = (T^j p)^{(k)}(0) = \langle T^k | T^j p(x) \rangle
    \end{align*}
\end{demo}

\begin{corolario}
    Sea $F(T) \in \F$ y $p(x) \in \poly$ cualesquiera. Entonces, se verifica que
        $$\begin{aligned}
            \langle F(T) | p'(x)\rangle = \langle T\cdot F(T)| p(x)\rangle
       \end{aligned}$$
    Equivalentemente, se verifica que $\langle F(T) | p(x)\rangle = \langle T\cdot F(T)| \int p(x)\rangle$
\end{corolario}

\begin{corolario}
    Dada $F(T) \in \F$ y $p(x) \in \poly$ se tiene que $ \langle  F(T) | p(x) \rangle = (F(T)p(x))(0)$.

    Es decir, se verifica que el diagrama
    \[\begin{tikzcd}
	{\poly} & {\poly}\\
	& {\field}
	\arrow["{F(T)}", from=1-1, to=1-2]
	\arrow["{\langle F(T) | \quad \rangle}"', from=1-1, to=2-2]
	\arrow["{\varepsilon_0}", from=1-2, to=2-2]
    \end{tikzcd}\]
    es un diagrama conmutativo.
\end{corolario}
Si bien la asignación \eqref{T como derivada} es biunívoca de forma trivial, a diferencia de lo que ocurría con el i.c.u, la aplicación $\varphi$ \underline{\textbf{no}} es una aplicación biyectiva; es decir, existen operadores que no provienen de ninguna serie formal. Esto es consecuencia de que, como $\gr(\partial p(x)) \leq \gr(p(x))$, por linealidad se tiene que
\begin{equation*}
    \gr(F(T)p(x)) \leq \gr(p(x))
\end{equation*}
es decir, que todos los operadores que son series formales actúan bajando o manteniendo el grado. Así pues, el operador multiplicación por $x$
\begin{equation*}
    p(x) \mapsto xp(x)
\end{equation*}
es un ejemplo de un operador que no proviene de ninguna serie formal.

\begin{obs}\label{multiplicacion por x operador}
    El operador \textit{multiplicación por $x$} nos aparecerá en repetidas ocasiones. En lo que sigue, lo escribiremos abreviadamente como $M$. Hacemos esto con el objetivo de evitar la posible confusión que podría crearse con la multiplicación de series formales y polinomios. En particular, con esta notación, la proposicón \ref{adjuncion M} se reescribe como
    \begin{equation*}
        \langle \partial F(T) | p(x) \rangle = \langle F(T) | M p(x)\rangle
    \end{equation*}
    de donde surge una nueva aparente relación de adjunción entre el operador $M$ y la derivada formal. Todo esto se esclarecerá en el capítulo tercero. 
\end{obs}

Mediante lo ya dicho, se ha probado la siguiente proposición.

\begin{proposicion}
    La aplicación $\varphi \colon \F \hookrightarrow \End(\poly)$ es un homomorfismo de anillos inyectivo pero no epiyectivo, con imagen
    \begin{equation*}
        \text{Im}(\varphi) = \{ \Lambda \in \End(\poly) : \exists F(T) \in \F \text{ con } \Lambda p(x) = F(T)p(x), \forall p(x) \in \poly \}
    \end{equation*}
\end{proposicion}

Así pues, nos interesa ahora dar una caracterización de la imagen del morfismo $\varphi$.

\begin{definicion}
    Diremos que un operador $L \in \End(\poly)$ tiene \textbf{forma de serie formal} si $L \in \text{Im}(\varphi)$.
\end{definicion}

\begin{definicion}
    Diremos que un operador $F(T) \in \text{Im}(\varphi)$ es un \textbf{operador delta} (o \textbf{$\delta$-operador}) si como serie formal $F(T)$ es una serie delta. Análogamente, diremos que $F(T)$ es un \textbf{operador invertible} si como serie formal $F(T)$ es una serie invertible. 
\end{definicion}

\begin{lema}
    Si $F(T)$ es un operador delta y $\Lambda \in \End(\poly)$ conmuta con $F(T)$, es decir, $F(T) \circ \Lambda = \Lambda \circ \F(T)$, entonces $\gr(\Lambda p(x)) \leq \gr(p(x))$ para todo $p(x) \in \poly$
\end{lema}
\begin{demo}
    Como venimos haciendo, por linealidad, basta con probarlo para los polinomios $x^n$. Por contradicción, supongamos que existe un número natural $m$ tal que $\gr(\Lambda x^n) \geq m+1$. Como $\ord(F(T)) = 1$, tenemos que, siendo $F(T) = \sum_{k \geq 1} \frac{a_k}{k!}T^k$
    \begin{align*}
        F(T)x^n & = \sum_{k=1}^n \binom{n}{k}a_k T^k = n a_1 x^{n-1} + \dots + a_n, \quad \forall n \geq 1\\
        F(T)x^{0} & = 0
    \end{align*} 
    Por tanto, $\gr(F(T)p(x)) = \gr(p(x)) - 1$ si $\gr(p(x)) \geq 1$. Luego, de aquí deducimos que, como $\ord(F^{m+1}(T)) = m+1$, entonces $F^{m+1}(T)x^m = 0$. Y, en consecuencia
    \begin{equation*}
        F^{m+1}(T) (\Lambda x^m) = \Lambda (F^{m+1}(T)x^m) = 0 
    \end{equation*} 
    Pero $\gr(\Lambda x^n) \geq m+1 \implies \gr(F^{m+1}(T) (\Lambda x^m)) \geq 0 \implies F^{m+1}(T) (\Lambda x^m) \neq 0$, lo cual es imposible. Con lo que $\gr(\Lambda x^n) \leq \gr(x^n)$ necesariamente y se concluye.
\end{demo}

\begin{teorema}\textbf{(caracterización Im($\varphi$))}
    Un operador $\Lambda \in \End(\poly)$ tiene forma de la serie formal $F(T)$ si y solo si conmuta con el operador derivada (i.e: $T \circ \Lambda = \Lambda \circ T$).
\end{teorema}

\begin{demo}
    Es claro que los elementos de $\text{Im}(\varphi)$ conmutan con $T$, puesto que conmutan con todas las series formales, luego en particular lo hacen con $T$. Recíprocamente, si $\Lambda$ conmuta con $T$, entonces $\Lambda$ tendrá la forma de $F(T)$ para la serie formal $F(T)$ definida como:
    \begin{equation*}
        F(T) = \sum_{k=0}^\infty \frac{\langle 1| \Lambda x^k\rangle}{k!}T^k
    \end{equation*}
    En efecto, ya que $\gr(\Lambda x^n) \leq n$, por el lema previo y la proposición \ref{taylor analogo} tenemos que
    \begin{equation*}
        \Lambda x^n = \sum_{k=0}^n \frac{\langle T^k| \Lambda x^n\rangle}{k!}x^k
    \end{equation*}
    Y utilizando las propiedades demostradas previamente junto a la hipótesis de conmutación con $T$, vemos que
    \begin{equation*}
        \langle T^k| \Lambda x^n\rangle = \langle 1| T^k\Lambda x^n\rangle = \langle 1| \Lambda T^k x^n\rangle = \langle 1| \Lambda (n)_k x^{n-k}\rangle
    \end{equation*}
    Por tanto, podemos reescribir $\Lambda x^n$ como
    \begin{equation*}
        \Lambda x^n = \sum_{k=0}^n \binom{n}{k}\langle 1| \Lambda x^{n-k}\rangle x^k = \sum_{k=0}^n \binom{n}{k}\langle 1| \Lambda x^{k}\rangle x^{n-k} = F(T)x^n 
    \end{equation*}
    donde en la segunda igualdad hemos usado que $\binom{n}{k} = \binom{n}{n-k}$
\end{demo}

\begin{corolario}
    Dado un operador delta $G(T)$, $\Lambda \in \End(\poly)$ tiene la forma de la serie formal $F(T)$ si y solo si conmuta con $G(T)$
\end{corolario}

\begin{obs}
    El operador delta $G(T)$ escogido puede ser cualquiera. Esta condición afirma que \underline{\textbf{tan solo es necesario}} comprobar que $\Lambda$ conmuta con este operador $G(T)$.
\end{obs}

\begin{demo}
    Si $\Lambda$ tiene forma de serie formal entonces conmuta con toda serie formal, en particular con $G(T)$. Recíprocamente, si $\Lambda \circ G(T) = G(T) \circ \Lambda$, como $G(T)$ es un operador delta, las series formales $G^k(T)$ forman una pseudobase de $\F$. En concreto, existen escalares $a_k \in \F$ tales que
    \begin{equation*}
        T = \sum_{k = 0}^\infty a_k G^k(T)
    \end{equation*}
    Por lo tanto, la conmutatividad implica que
    \begin{align*}
        \Lambda T x^n & = \Lambda(\sum_{k=0}^\infty a_k G^k(T) x^n) = \Lambda (\sum_{k=0}^n a_k G^k(T) x^n) = \sum_{k=0}^n a_k \Lambda G^k(T) x^n\\
        & = \sum_{k=0}^n a_k G^k(T) \Lambda x^n = \sum_{k=0}^\infty a_k G^k(T) \Lambda x^n = T \Lambda x^n
    \end{align*}  
    donde la penúltima igualdad se debe a que $\gr(\Lambda x^n) \leq n$ por el lema previo.
\end{demo}

\begin{corolario}
    $\Lambda \in \End(\poly)$ tiene forma de serie formal si y solo si conmuta con el operador de traslación $e^{\lambda T} \quad (\lambda \neq 0)$
\end{corolario}

\begin{demo}
    $\Lambda$ conmuta con $e^{\lambda T} \iff \Lambda$ conmuta con el operador diferencia $e^{\lambda T} - 1$. Como el operador diferencia es un operador delta siempre que $\lambda \neq 0$, por el corolario anterior se concluye.   
\end{demo}

Por otro lado, conviene que dotemos al espacio de operadores de una estructura, además, de espacio topológico que haga que la aplicación $\varphi$ sea continua, también de una forma similar a como lo hacíamos con los funcionales. Esta topología se corresponde nuevamente con la topología de estabilización. La demostraciones son casi idénticas y quedan omitidas.

\begin{proposicion}
    Con la topología de estabilización y la estructura de álgebra usual, el espacio de operadores $\End(\poly)$ es un álgebra topológica completa.
\end{proposicion}

\begin{proposicion}
    Con la topología formal y la topología de estabilización, la aplicación $\varphi$, que asigna a cada serie formal su operador asociado, es continua.
\end{proposicion}

\begin{proposicion}
    La topología de estabilización en $\End(\poly)$ se corresponde con la topología inicial de la aplicación $\varphi$.
\end{proposicion}

Recuperamos ahora los ejemplos de funcionales dados con anterioridad con el objetivo de dar su intepretación como operadores, cerrando así esta sección y, con ello, este primer capítulo.

\begin{ej}
    \begin{enumerate}
        \item El \textbf{operador derivada $k$-ésima} $T^k$ viene dado por \ref{T como derivada} y para un polinomio genérico $p(x) \in \poly$ será
        \begin{equation} \label{operador derivada}
            T^k p(x) = p^{(k)}(x)
        \end{equation}
        \item El \textbf{operador traslación} $e^{\lambda T}$ cumple que
        \begin{equation*}
            e^{\lambda T} x^n = \sum_{k=0}^\infty \frac{\lambda^k}{k!}x^n = \sum_{k=0}^n \binom{n}{k}\lambda^k x^{n-k} = (x+ \lambda )^n
        \end{equation*} 
        Con lo que, para un polinomio genérico $p(x)$
        \begin{equation} \label{operador traslacion}
            e^{\lambda T} p(x) = p(x+\lambda)
        \end{equation}
        \item El \textbf{operador diferencia} $e^{\lambda T} -1$ en este caso será
        \begin{equation}\label{operador diferencia}
            (e^{\lambda T} -1) p(x) = p(x+ \lambda) - p(x)
        \end{equation}
        \item El \textbf{operador de Abel} $Te^{\lambda T}$ será, por el producto formal
        \begin{equation}\label{operador de Abel}
            Te^{\lambda T}p(x) = p'(x+ \lambda)
        \end{equation}
        \item El operador $\frac{e^{\lambda T} - 1}{T}$ verifica que
        \begin{align*}
            \frac{e^{\lambda T} - 1}{T}x^n & = (\sum_{k=0}^\infty \frac{\lambda^{k+1}}{(k+1)!}T^k )x^n = \sum_{k=0}^n \binom{n+1}{k+1} \frac{\lambda^{k+1}}{(n+1)}x^{n-k}\\
            & = \frac{1}{n+1}\sum_{k=1}^{n+1} \binom{n+1}{k} \lambda^k x^{n+1-k} = \frac{1}{n+1}((x+ \lambda )^{n+1} - x^{n+1}) = \int\limits_{x}^{x+ \lambda} u^n \, du
        \end{align*}
        Por tanto, para un polinomio genérico $p(x)$
        \begin{equation}\label{operador integral media}
            \frac{e^{\lambda T} - 1}{T}p(x) = \int\limits_{x}^{x+ \lambda} p(u) \, du
        \end{equation}
    \end{enumerate}
\end{ej}
%%%%%%%%%%%%%%%%%%%%%
%%%%%%%%%%%%%%%%%%%%%
%%%%%%%%%%%%%%%%%%%%%
%%%%%%%%%%%%%%%%%%%%%
%%%%%%%%%%%%%%%%%%%%%

%: Página vacía
\newpage
\hfill
\thispagestyle{empty} % Eliminar numeración de la página

\include{Sucesiones de Sheffer/Sucesiones de Sheffer}

%: Página vacía
\newpage
\hfill
\thispagestyle{empty} % Eliminar numeración de la página

\chapter{Operadores en el espacio dual de polinomios}

%: Página vacía
\newpage
\hfill
\thispagestyle{empty} % Eliminar numeración de la página

\chapter{Aplicaciones del cálculo umbral}

%: BERNOULLI
\section{Polinomios de Bernoulli}

%: Apéndices
\appendix % Inicio sección de apéndices

\chapter{Series formales} \label{series formales} % Crea el título sin numerar "Apéndices"


%: SERIES FORMALES
%%%%%%%%%%%%%%%%%%
%%%%%%%%%%%%%%%%%%
%%%%%%%%%%%%%%%%%%
%%%%%%%%%%%%%%%%%%

Toda la información de esta sección ha sido extraída de \cite{sambale}. Damos aquí las definiciones y los resultados de los que nos servimos a lo largo de todo el trabajo.

\begin{definicion}
    Sea $\mathcal{R}$ un dominio íntegro y sea $\alpha$ una sucesión infinita en $\mathcal{R}$, es decir:
    \begin{equation*}
        \alpha = \{a_0, a_1, a_2, \dots \} \hspace{0.25cm} a_n \in \mathcal{R}, \forall n \in \N 
    \end{equation*}
    Se define y simboliza con $\F$ al conjunto de estas sucesiones. Se dice que dos elementos $\alpha = \{a_0, a_1, a_2, \dots \}$ y $\beta = \{b_0, b_1, b_2, \dots \} $ de $\F$ son iguales si y solo si $a_n = b_n$ para todo $n$. 
\end{definicion}
Podemos dotar a este conjunto de una estructura algebraica mediante las operaciones de suma y producto definidas como:
\begin{equation*}
    \alpha + \beta \equiv \{a_0 + b_0, a_1 + b_1, a_2 + b_2, \dots \}
\end{equation*}
\begin{equation*}
    \alpha \cdot \beta \equiv \{a_0  b_0, a_0b_1 + a_1b_0, a_0b_2 + a_1b_1 + a_2b_0, \dots, \sum_{k=0}^na_kb_{n-k}, \dots\}
\end{equation*}

\begin{teorema}
    $(\F, +, \cdot)$ es un anillo. El elemento neutro y la unidad son, respectivamente:
    \begin{equation*}
        0 \equiv \{0,0,0,\dots\} \hspace{2cm} 1\equiv \{1, 0, 0, \dots\}
    \end{equation*}
    y el elemento opuesto o inverso respecto a la suma es:
    \begin{equation*}
        -\alpha \equiv \{-a_0, -a_1, -a_2, \dots \}
    \end{equation*}
\end{teorema}

\begin{definicion}
    Se definen los subconjuntos $\F_0$ y $\F_1$ de $\F$ como:
    \begin{equation*}
        \F_0 \equiv \{ \alpha \in \F : a_0 = 0 \}
    \end{equation*}
    \begin{equation*}
        \F_1 \equiv \{ \alpha \in \F : a_0 = 1 \}
    \end{equation*}
\end{definicion}

\begin{teorema}
    \textbf{(existencia inverso multiplicativo)} Un elemento $\alpha = \{a_0, a_1, a_2, \dots\} \in \F$ tiene inverso multiplicativo si y solo si $a_0 \in \mathcal{R}^*$. En particular, si $\mathcal{R}$ es un cuerpo, $\alpha$ tiene inverso multiplicativo si y solo si $\alpha \not \in \F_0$ 
\end{teorema}

\begin{proposicion}
    Dado un entero positivo $n$ y $\alpha \in \F_1$, se verifica que $(\alpha^{-1})^n = (\alpha^n)^{-1}$ 
\end{proposicion}

\begin{definicion}
    Se llama \textbf{anillo de series formales en la variable $T$ con coeficientes en $\mathcal{R}$} al conjunto $\mathcal{R}[[T]]$ definido como:
     \begin{equation*}
         \mathcal{R}[[T]] \equiv \{ F(T) = \sum_{n \geq 0} a_nT^n : a_n \in \mathcal{R}\}
    \end{equation*}
    dotado de las operaciones de suma y producto dadas por:
   \begin{equation*}
    F(T) + G(T) \equiv \sum_{n\geq 0} (a_n + b_n)T^n  
    \end{equation*}
    \begin{equation*}
    F(T) \cdot G(T) \equiv \sum_{n \geq 0}c_nT^n \text{ donde } c_n=\sum_{k=0}^na_nb_{n-k}
    \end{equation*}
    En concreto, a la variable $T$ se la denomina la \textbf{variable formal}
\end{definicion}

Con las definiciones dadas para la suma y el producto es natural que pensemos y escribamos los elementos de $\F$ como elementos de $\mathcal{R}[[T]]$:
\begin{equation*}
    \alpha = \{a_0, a_1, a_2, \dots \} \iff \alpha \equiv F(T) = \sum_{n=0}^\infty a_nT^n
\end{equation*}
ya que el producto y suma de los factores de $\alpha$ coincide con el producto y suma de cada factor de la serie formal $F(T)$

%%%%%%%%%%%%%%%%%%
%%%%%%%%%%%%%%%%%%
%%%%%%%%%%%%%%%%%%
%%%%%%%%%%%%%%%%%%
%%%%%%%%%%%%%%%%%%

\begin{thebibliography}{99}

\bibitem{guinand}
Guinand, A. P.
\textit{The Umbral Method: A Survey of Elementary Mnemonic and Manipulative Uses},  
The American Mathematical Monthly, 86(3), 187–195 (1979)
\url{https://doi.org/10.1080/00029890.1979.11994763}

\bibitem{bucchianico}
Di Bucchianico, A.; Loeb, D.  
\textit{A selected survey of umbral calculus}, 
The Electronic Journal of Combinatorics, 1, Article DS3 (1995)

\bibitem{roman1}
Roman, Steven, 
\textit{The umbral calculus},
Pure and Applied Mathematics, vol. 111,
London: Academic Press Inc (1984)

\bibitem{roman2}
Roman, Steven M.; Rota, Gian-Carlo,
\textit{The umbral calculus}, Advances in
Mathematics, 27 (2): 95–188,
(1978)
\url{https://doi.org/10.1016/0001-8708(78)90087-7}

\bibitem{rota}
G.-C. Rota; B.D. Taylor. 
\textit{The classical umbral calculus}, 
SIAM J. Math. Anal., 25:694–711 (1994)

\bibitem{bell}
Bell, E. T., 
\textit{The History of Blissard's
Symbolic Method, with a Sketch of its
Inventor's Life}, 
The American Mathematical Monthly, 45 (7),
Mathematical Association of America: 414–
421 (1938)
\url{https://doi.org/10.1080/00029890.1938.11990829}

\bibitem{kim}
Dae San Kim; Taekyun Kim,
\textit{Umbral calculus associated with Bernoulli polynomials},
Journal of Number Theory, Volume 147, Pages 871-882, ISSN 0022-314X (2015)
\url{https://doi.org/10.1016/j.jnt.2013.09.013}

\bibitem{blissard}
John Blissard, 
\textit{Theory of Generic Equations}, 
Quarterly Journal of Pure and Applied Mathematics, vol. 5, pp. 58-75, 185-208 (1862)

\bibitem{verdoodt}
A. Verdoodt, 
\textit{p-adic q-umbral calculus}, 
J. Math. Ann. Appl., 198:166–177 (1996)

\bibitem{sambale}
Sambale, B., 
\textit{An Invitation to Formal Power Series}, 
Jahresber. Dtsch. Math. Ver. 125, 3–69 (2023) \url{https://doi.org/10.1365/s13291-022-00256-6}

\end{thebibliography}

\end{document}
